%%%%%%%%%%%%%%%%%%%%%%%%%%%%%%%%%%%%%%%%%
% Engineering Calculation Paper
% LaTeX Template
% Version 1.0 (20/1/13)
%
% This template has been downloaded from:
% http://www.LaTeXTemplates.com
%
% Original author:
% Dmitry Volynkin (dim_voly@yahoo.com.au)
%
% License:
% CC BY-NC-SA 3.0 (http://creativecommons.org/licenses/by-nc-sa/3.0/)
%
%%%%%%%%%%%%%%%%%%%%%%%%%%%%%%%%%%%%%%%%%

%----------------------------------------------------------------------------------------
%	PACKAGES AND OTHER DOCUMENT CONFIGURATIONS
%----------------------------------------------------------------------------------------

\documentclass[12pt,a4paper]{article} % Use A4 paper with a 12pt font size - different paper sizes will require manual recalculation of page margins and border positions

\usepackage{marginnote} % Required for margin notes
%\usepackage{wallpaper} % Required to set each page to have a background
%\usepackage{lastpage} % Required to print the total number of pages
\usepackage[left=1.3cm,right=4.6cm,top=1.8cm,bottom=4.0cm,marginparwidth=3.4cm]{geometry} % Adjust page margins
\usepackage{amsmath} % Required for equation customization
\usepackage{amssymb} % Required to include mathematical symbols
\usepackage{xcolor} % Required to specify colors by name

\usepackage{fancyhdr} % Required to customize headers
\setlength{\headheight}{80pt} % Increase the size of the header to accommodate meta-information
\pagestyle{fancy}\fancyhf{} % Use the custom header specified below
\renewcommand{\headrulewidth}{0pt} % Remove the default horizontal rule under the header

\setlength{\parindent}{0cm} % Remove paragraph indentation
\newcommand{\tab}{\hspace*{2em}} % Defines a new command for some horizontal space

\newcommand\BackgroundStructure{ % Command to specify the background of each page
\setlength{\unitlength}{1mm} % Set the unit length to millimeters



\setlength\fboxsep{0mm} % Adjusts the distance between the frameboxes and the borderlines
\setlength\fboxrule{0.5mm} % Increase the thickness of the border line
\put(10, 10){\fcolorbox{black}{blue!10}{\framebox(155,247){}}} % Main content box
\put(165, 10){\fcolorbox{black}{blue!10}{\framebox(37,247){}}} % Margin box
\put(10, 262){\fcolorbox{black}{white!10}{\framebox(192, 25){}}} % Header box
%\put(137, 263){\includegraphics[height=23mm,keepaspectratio]{logo}} % Logo box - maximum height/width: 
}

%----------------------------------------------------------------------------------------
%	HEADER INFORMATION
%----------------------------------------------------------------------------------------

\fancyhead[L]{\begin{tabular}{l r | l r} % The header is a table with 4 columns
\textbf{Project} & Cubic Galileon Radiation of Non-Relativistic Masses perturbed in Time & \textbf{Page} & \thepage/\pageref{LastPage} \\ % Project name and page count
%\textbf{Job} & 0001 & \textbf{Updated} & 27/11/2012 \\ % Job number and last updated date
%\textbf{Version} & Design-1A-RC-001 & \textbf{Reviewed} & 2/12/2012 \\ % Version and reviewed date
\textbf{Student} & Scott Perkins & \textbf{Reviewer} & Hector Silva and Nico Yunes \\ % Designer and reviewer
\end{tabular}}

%----------------------------------------------------------------------------------------

\begin{document}

%\AddToShipoutPicture{\BackgroundStructure} % Set the background of each page to that specified above in the header information section

%----------------------------------------------------------------------------------------
%	DOCUMENT CONTENT
%----------------------------------------------------------------------------------------

\section{Background and General Info for Cubic Galileon and the Project} 

The Cubic Galileon has an action given by equation \ref{eq:1}
\begin{equation}\label{eq:1}
\begin{split}
S = \int d^4x \sqrt{-g} [\frac{R}{2}- \frac{1}{2} \phi_{,\mu} \phi^{,\mu} - \frac{1}{\Lambda^2} \phi_{,\mu} \phi^{,\mu} \Box \phi] + S_m[\tilde{g_{\mu \nu}}], \\
& \tilde{g_{\mu \nu}} = e^{2 \alpha \phi} g_{\mu \nu}
\end{split}
\end{equation}
 
Equation 1 results in the equations of motion for the field given by equation \ref{eq:2}
\begin{equation}\label{eq:2}
\Box \phi + \frac{2}{\Lambda^2} [ (\Box \phi)^2 -\nabla_{\mu}\nabla_{\nu}\phi\nabla^{\mu}\nabla^{\nu}\phi] = 8 \pi \alpha \rho
\end{equation}

First, we will solve equation \ref{eq:2} for a static and spherically symmetric case, giving us the scalar field for a nonrelativistic, stationary mass. We will then take this solution and perturb it temporally, which will give us a small perturbation to the field that will tell us about the possible effects of scalar radiation. 
\subsection{Spherically Symmetric and Static Assumptions}
Assuming spherical symmetry and a static metric, all derivatives besides the radial derivative will vanish, resulting in an simplifications given in equation \ref{eq:4}

\begin{equation}\label{eq:4}
\begin{split}
\Box \phi = \phi_{,r,r} + \frac{2}{r}\phi_{,r} \\
& \nabla_{\mu}\nabla_{\nu}\phi\nabla^{\mu}\nabla^{\nu}\phi = \phi_{,r,r}^2 + \frac{2}{r} \phi_{,r}^2
\end{split}
\end{equation}
These simplifications result in the equation of motion for the field in equation \ref{eq:3}.
\begin{equation}\label{eq:3}
\phi_[,r,r] + \frac{2}{r} \phi_{,r} + \frac{8}{\Lambda^2 r}\phi_{,r} + \frac{4}{\Lambda^2 r^2}\phi_{,r}^2 = 8 \pi \alpha \rho
\end{equation}

The solution to this equation will set the stage for our perturbed system, giving us a background solution to overlay our time dependent perturbation. The solution was determined by the shooting method, implemented in Mathematica, and using the initial conditions shown in equation \ref{eq:5}
\begin{equation}\label{eq:5}
\begin{split}
\phi_{,r}(0) = 0 \\
&\phi_{\infty} = \phi_{cosmological}\\
& \phi_{<}(r=R) = \phi_{>}(r=R)\\
&\phi'_{<}(r=R) = \phi'_{>}(r=R)
\end{split}
\end{equation}
where $\phi_{cosmological}$ is the value of the field in vacuum far from any sources, inspired by cosmological measurements. 
\subsection{Time Perturbation}

The field and source were then perturbed temporally, resulting in a new field and source given in equation \ref{eq:6}
\begin{equation}\label{eq:6}
\begin{split}
\phi \Rightarrow \phi_0 + \delta\phi \\
&\rho \Rightarrow \rho_0 + \delta \rho sin(\omega t)
\end{split}
\end{equation}
Incorporating time in the equations of motion for the field result in new operators defined in equation \ref{eq:7}
\begin{equation}\label{eq:7}
\begin{split}
\Box \phi = \phi_{,r,r} + \frac{2}{r}\phi_{,r} - \phi_{,t,t} \\
& \nabla_{\mu}\nabla_{\nu}\phi\nabla^{\mu}\nabla^{\nu}\phi = \phi_{,r,r}^2 + \frac{2}{r} \phi_{,r}^2
\end{split}
\end{equation}
Equation\ref{eq:7} gives a new equation of motion, shown in equation \ref{eq:8}
\begin{equation}\label{eq:8}
\phi_{,r,r} + \frac{2}{r} \phi_{,r} -\phi_{,t,t}+ \frac{2}{\Lambda^2 }[\frac{4}{r}\phi_{,r}\phi_{r,r} - \phi_{r,r}\phi_{,t,t}-\frac{4}{r}\phi{,t,t}\phi_{,r} + \frac{2}{r^2}\phi_{,r}^2] = 8 \pi \alpha \rho(r,t)
\end{equation}
Because we are still assuming flat and static space (ie $\eta_{\mu \nu}$), the Christoffel symbols for the system are still time independent, resulting only in additional terms $\propto \phi_{,t,t}$. Inserting the definitions of $\phi$ and $\rho$ from equation \ref{eq:6} into equation \ref{eq:8}, then expanding and dropping terms $\sim \delta \phi^2$ gives us the equation of motion for the perturbation to the field:
\begin{equation}\label{eq:9}
\delta\phi_{,r,r} + \frac{2}{r} \delta\phi_{,r} -\delta\phi_{,t,t}+ \frac{2}{\Lambda^2 }[\frac{4}{r}\delta\phi_{,r}\phi_{r,r} +\frac{4}{r}\delta\phi_{,r,r}\phi_{,r}-\phi_{r,r}\delta\phi_{,t,t}-\frac{4}{r}\delta\phi_{,t,t}\phi_{,r} + \frac{2}{r^2}\delta\phi_{,r}^2] = 8 \pi \alpha \delta\rho sin(\omega t)
\end{equation}

\subsection{Computation}

%----------------------------------------------------------------------------------------

\end{document}